\section*{Zusammenfassung}
Walid Maalej und Martin P. Robillard haben im September 2013 einen Artikel \cite{MaalejRobillard} ver�ffentlicht, in dem sie die Dokumentationen der Programmiersprachen Java und .NET auf ihren Informationsgehalt hin untersucht und verglichen haben. Ihre Untersuchung wird im Rahmen dieser Bachelorarbeit auf die Dokumentation von \href{https://www.python.org/}{Python}\footnote{https://www.python.org/} mit einigen Abweichungen �bertragen. \par
Die von Maalej und Robillard eingef�hrte Taxonomie der in Dokumentationen anzutreffenden Wissenstypen wurde hierf�r auf die Eigenheiten von Python angepasst. Die Einordnung von Teilen der Dokumentationen zu den verschiedenen Wissenstypen wird von Gutachtern im Rahmen eines Forschungspraktikums geleistet und erfolgt mit Hilfe eines eigens hierf�r geschriebenen Werkzeuges. \par
Die anschlie�ende Auswertung der Ergebnisse und ein Vergleich dieser mit denen aus dem Originalartikel zeigt, dass die vorgenommenen �nderungen in der Konzeption sinnvoll waren. Die Struktur der Wissenstypen in der Dokumentation von Python �hnelt stark der bei Java und .NET, hat aber auch �berraschende Abweichungen. Die erhobenen Daten bieten zudem eine Grundlage f�r weitergehende Untersuchungen zu den Informationstypen in der Python-Dokumentation wie beispielsweise die Reihenfolge des Auftretens oder die Wortmenge von Informationen. 


