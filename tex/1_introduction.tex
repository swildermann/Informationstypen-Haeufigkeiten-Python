\section{Einleitung}
Zu jeder Programmiersprache und Programmierschnittstelle geh�rt in der Regel eine Dokumentation �ber die bereitgestellten Funktionalit�ten, auch Referenzhandbuch genannt. 
W�hrend die Vor- und Nachteile der Programmiersprachen h�ufig disktutiert werden, findet man nur wenig Analyse zu den Dokumentationen.
Dabei tr�gt eine Dokumentation nicht unwesentlich zur Qualit�t einer Programmiersprache bei. Entscheidend ist, wie gut 
die Entwickler mit den gebotenen Informationen zurechtkommen und, wie schnell Antworten auf Benutzungsfragen gefunden werden k�nnen. 
Die Anzahl und Qualit�t der Programmbeispiele ist dabei mindestens genauso wichtig wie die Erl�uterung von Funktionalit�ten, Konzepten und Abh�ngigkeiten. 
\par
Um etwas �ber die Qualit�t von Dokumentationen aussagen zu k�nnen, muss erst einmal verstanden werden, welche Informationen zu welchen Teilen in dem jeweiligen Handbuch vorhanden sind. Der erste Teil dieser Frage wurde bereits von Walid Maalej und Martin Robillard \cite{MaalejRobillard} beantwortet. Sie fanden bei der Analyse der JAVA und .NET Dokumentationen insgesamt zw�lf gut unterscheidbare Wissenstypen, im Original hei�en diese "`knowledge types"'.
\par
Die Analyse �ber die Verteilung dieser Typen in der offiziellen Python-Dokumentation (Version 3.4.0) wird von Studenten innerhalb eines Forschungspraktikums am Institut f�r Informatik durchgef�hrt. 
Sie erhalten Ausschnitte aus der Dokumentation �ber eine hierf�r entwickelte Website und geben dann an, welche der zw�lf Typen auf diese Einheit passen. 
Die genauen Regeln f�r die Bewertung dieser Einheiten gibt das Kodier-Handbuch an, welches im Wesentlichen aus der Originalstudie �bernommen und auf Python angepasst wurde.

\subsection{Aufbau der Arbeit}
Am Beginn stelle ich die vorausgegangene Arbeit von Walid Maalej und Martin P. Robillard \cite{MaalejRobillard} vor, da diese sie Grundlage f�r diese Bachelorarbeit bildet. Hierbei gehe ich insbesondere auf die verschiedenen Informationstypen (auch Wissenstypen genannt) ein. Im Anschluss erkl�re ich, welche �nderungen an diesen Wissenstypen notwendig waren, um auf die Analyse mit Python zu passen. 
Die Programmierung des Werkzeugs f�r die Typisierung der Dokumentationseinheiten wird dann ebenso erl�utert wie die Begr�ndung f�r eine eigene Entwicklung zu diesem Zweck.
Die Durchf�hrung und Organisation inklusive der aufgetretenen Schwierigkeiten des Forschungspraktikums, innerhalb dessen die Studenten eine bestimmte Stichprobe an Dokumentationseinheiten erhielten, um diese zu typisieren, wird ebenso thematisiert. 
Zuletzt werden dann die ausgewerteten Ergebnisse vorgestellt und es wird ein Fazit gezogen. 