\section{Extrahierer}

Um die Dokumentationseinheiten aus der HTML-Dokumentation von Python zu bekommen, war es nötig, einen Extrahierer zu schreiben. Dieser wurde in Python mit Hilfe von BeautifulSoup4 angefertigt. Entsprechend der Definitionen von Dokumentationseinheiten sollten also die HTML-Schnipsel getrennt von einander in die Datenbank einfließen. Diese Einheiten sind allerdings häufig geschachtelt, so dass eine Methodendeklaration in der Regel innerhalb einer Klasenbeschreibung vorkommt, welche wiederrum innerhalb einer Sektion anzutreffen ist. Um Dopplungen bei der Typisierung zu vermeiden, habe ich deswegen Platzhalter der Form `[something removed here] an solchen Stellen eingebaut. Platzhalter haben einen Vorteil gegenüber dem einfachen Wegelassen dieser Elemente, denn so sieht auch der Gutachter, dass hier etwas von der Original-Dokumentation abweicht und kann sich somit die entstandenen Lücken erklären. Dies tritt besonders häufig bei Sektionen auf, da diese alle weiteren Elemente beinhalten. 
