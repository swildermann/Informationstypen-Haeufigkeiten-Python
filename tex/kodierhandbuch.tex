\begin{shaded}
You will be presented with documentation blocks extracted from API reference documentation (Javadocs and the like). For each block, you will be also presented with the name of its corresponding package/namespace, class, method, or field. Your task is to read each block carefully and evaluate where the block contains knowledge of the different types described below. Apply the following rules when doing so:
\begin{itemize}
\item Consider the documentation initially one paragraph at a time. If the paragraph contains only information of one knowledge type, mark the whole paragraph with that type in one stretch.
Never mark more than one paragraph at once.
\item If multiple knowledge types mix within the paragraph, mark a contiguous stretch of one or more sentences with one type and the next stretch with another.
\item If necessary, treat subsentences connected with conjunctions such as "`and"', "`or"', "`but"', or with colon or semicolon like complete sentences.
\item A sentence (or such subsentence) as a whole is never marked with more than one type, but sometimes phrases within the sentence will require a separate marking with a different type. Double-marking the same text with two types is allowed (and required) in this case. To create such annotations uniformly, we work in two passes:
\begin{itemize}
\item Pass 1: Prefer longer segments of a complete sentence or several. Annotate subsentences only rarely.
If a sentence contains knowledge of more than one type (which happens quite often), look if one of them is clearly dominant for the overall role of the sentence in the documentation block. If so, annotate only that dominant type to the whole sentence and do not annotate any of the other types yet.
\item Pass 2: After pass 1, many relevant annotations will be missing. We now add those on top of the pass 1 annotations as double annotations. For the double annotations, we still prefer complete subsentences where possible (or other clearly delineated parts such as parentheses), but choose the shorter of two possibilities whenever we are unsure.
\end{itemize}
\item Rate the knowledge type as true only if there is clear evidence that knowledge of that type is present in the stretch.
If you have doubts, consult the type's definition below.
If the doubts do not disappear, do not annotate that type.
\item However, all text of the documentation must be marked with a type. (Only hand-written documentation, not the signature itself and not the placeholders [Something removed here] that indicate left-out nested documentation blocks).
\end{itemize}
Read (and re-read whenever needed) the following descriptions very carefully. They explain how to recognize each knowledge type.
\end{shaded}