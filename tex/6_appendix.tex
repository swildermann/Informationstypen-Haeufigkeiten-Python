\section{Anhang}
\subsection{Quelltext}
Die entwickelten Werkzeuge (Website, Extrahierer, Django-Kommandos) sind alle schon w�hrend der Entwicklung in dieses Repository auf bitbucket.org "`gepusht"'\footnote{Befehl des Versionskontrollsystems "`GIT"' f�r das Hochladen von �nderungen} worden: https://bitbucket.org/svenwildermann/python-doc-application .
Der letzte Commit\footnote{Befehl des Versionskontrollsystems "`GIT"' f�r das Bekanntgeben von �nderungen} fand am <DATUM> statt und besitzt folgenden HASH-Wert, wobei die ersten sieben Zeichen bereits f�r eine Identifizierung gen�gen.
\begin{lstlisting}
941cff281a59907164824d61dc912803f1b11a64
\end{lstlisting}
Die Angabe dieses Wertes best�tigt den Stand der Programmierung zum Zeitpunkt der Abgabe. Weiterentwicklungen nach Abgabe der Bachelorarbeit sind, speziell f�r weiterf�hrende Auswertungen �ber den Rahmen dieser Arbeit hinaus, m�glich und ggf. auch n�tig. 

\subsection{Technologien}
\subsubsection{Django}
\subsubsection{Python}
\subsubsection{BeautifulSoup4}
\subsubsection{Coffeescript}
\subsubsection{Postgres}
\subsubsection{Ajax}
\subsection{Kodierhandbuch -? vollst�ndig}
\subsection{Glossar}
\begin{enumerate}
\item knowledge type/Informationstypen
\item API - in Implementierung des Extrahierer
\item DOM-Element

\end{enumerate}
