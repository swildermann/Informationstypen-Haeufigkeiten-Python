\section{Fazit}
Das erf�llte Ziel dieser Arbeit war das �bertragen der Informationtypen-Analyse von Walid Maalej und Martin Robillard \cite{MaalejRobillard} auf die Python-Dokumentation.
Mit Hilfe der erhobenen Daten und der dar�ber ausgef�hrten Auswerten k�nnen nun fundamentierte Aussagen �ber die Art und Verteilung von Informationen in der Dokumentation getroffen werden. Au�erdem k�nnen diese Daten mit den Ergebnissen f�r die Dokumentationen in JAVA und .NET verglichen werden. 

\subsection{Vergleich}
Es ist zu beachten, dass es teilweise Unterschiede in den Berechnungen zwischen dieser Arbeit und der Originalstudie gibt. 
Da in dieser Arbeit die Wissenstypen nicht �ber das Setzen von Haken in "`CheckBoxes"' sondern �ber das Markieren von Zeichenketten geschieht, sind die Ergebnisse nicht nur detailreicher sondern auch weniger zuf�llig. Daher entfallen die Auswertungen dar�ber, wie wahrscheinlich es ist, dass sich zwei Gutachter zuf�llig einig sind. 
\\
Dennoch gibt es einige Auswertungen, die sowohl in dieser Arbeit als auch in der anderen Studie vorgenommen wurden. Die Berechnung der �bereinstimmung von Typisierungen verschiedener Gutachter wurde analog ausgef�hrt und zeigt, dass in dieser Arbeit die �bereinstimmungen zwischen den Gutachtern im Median um fast sechs Prozent h�her sind (82 Prozent zu 87,9 Prozent). Die Werte f�r die minimale und die maximale �bereinstimmung zwischen zwei Gutachtern sind sogar fast zehn Prozent h�her. Das f�hrt zu dem Schluss, dass die in dieser Arbeit gew�hlte Methode zur Erfassung von Informationstypen zu gr��erer Sorgfalt gef�hrt hat, obgleich die Dauer f�r die Typisierung pro Einheit anstieg. \\
Die Berechnungen f�r die Unstimmigkeiten zwischen den Gutachtern sind in dieser Arbeit erheblich anders durchgef�hrt worden als in der Originalstudie und bieten somit keine direkte Vergleichsm�glichkeit. Man erkennt aber sehr wohl, dass es bei beiden Studien nur wenig Unstimmigkeiten bez�glich des Wissenstypen "`Code Examples"' gibt und eher viel Uneinigkeit dar�ber, wann die Wissenstypen "`Structure and Relationship"' und "'Functionality and Behaivor"' zutreffend sind. \\
Die Verteilung der verschiedenen Wissenstypen �ber alle Einheiten zeigt einige Parallelen. So ist sind allen drei Dokumentationen (Java, .NET, Python) viel Einheiten mit dem Wissenstyp "`Functionality and Behaivor"' vorhanden.  Zu etwa 20 Prozent kommt in allen untersuchten Dokumentationen auch der Wissenstyp "`Structure and Relationship"' vor. Einen entscheidenden Unterschied gibt es bei der Menge an Einheiten mit dem Wissenstyp "`Non-Information"'. W�hrend bei Java und .NET in etwa der H�lfte der Dokumentationseinheiten dieser Typ identifiziert wurde, waren es bei Python nur etwa sechs Prozent. \\
Wenn man die Wissenstypen aufgeteilt nach Kategorie beider Studien vergleicht, erkennt man bei Java und Python den selben Wert f�r die Kategorie "`Klassen"`und den Wissenstyp "`Functionality and Behaivor"'. In beide Dokumentationen liegt der Wert bei 64 Prozent. Fast identisch sind zudem die Werte dieser Kategorie f�r die Typen "`Structure and Relationship"`und "`Directives"'. Gleichzeitig findet man in etwa jeder zweiten Methodenbeschreibung in JAVA und .NET "`Non-Information"'. Bei Python ist dies gerade einmal in jeder f�nfzigsten Methode der Fall. \\
W�hrend bei JAVA und .NET sowohl positive als auch negative Korrelationen berechnet wurden, ist es in dieser Studie bei der Berechnung der positiven Korrelationen geblieben. 

\subsection{Weitere Auswertungen}
-�berlappende Markierungen gesondert betrachten 
-Reihenfolge von Markierungen
-Menge von Markierungen
-So einmal JAVA und .NET bewerten und vergleichen



