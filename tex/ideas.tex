\section*{Offtopic: Verbesserungen f�r die Bachelorarbeit}

\begin{enumerate}
\item \sout{Ein DOM-Knotenbaum einer typischen HTML-Datei einbauen}
\item \sout{UML-Diagramm des Tools} ergibt wenig Sinn wegen Django
\item \sout{Datenbank-Eintit�ten aufzeigen} Nur wenn noch Zeit �brig ist
\item \sout{M�glicher Workflow (als Nicht-deterministischer Automat oder so)} daf�r nicht komplex genug
\item \sout{Screenshot z.B. von Dokumentationseinheit1898 hinzuf�gen, um das Interface zu zeigen}
\item \sout{Alle Links als Fussnoten anzeigen - sieht besser aus und ist besser f�r den offline-Gebrauch!} 
\item \sout{Nummerierung bei den Code-Schnipseln hinzuf�gen}
\item \sout{Bei Bildern immer Bildunterschriften einf�gen, die etwas zu dem gezeigten aussagen}
\item Alle Kommentare im Code (und den code selbst) auf Rechtschreibfehler �berpr�fen
\item Related Work 
\item Future Work
\item \sout{Wort "Klassifizierung" statt " Typisierung"} Es bleibt bei Typisierung
\item \sout{Links im Literaturverzeichnis anzeigen lassen}
\item HTML-Fehler in der Original-Dokumentation aufzeigen (mindestens 2)
\item HASH-Wert des Quellcodes abdrucken
\item \sout{Alle geschriebenen Kommandos erw�hnen?} Neee zu viel
\item \sout{Stichprobe von Robert erw�hnen} 
\item Passiv vermeiden! 
\item Zusammenfassung: Auf Resultate eingehen (daf�r aber nicht auf bs4 etc.)
\item Mit der Goldstrichprobe vergewissern: Wie oft waren die Studenten sich f�lschlicherweise einig? Damit findet man einen stochastischen Fehler.
\item Umstimmigkeiten nach Kategorien?! 
\item Bild von dem Cado-Tool rein stellen (damit der Vergleich mit dem jetztigen Tool besser gelingt)
\item Kookurrenz noch einmal �berdenken: Ist das so richtig berechnet? Durch den falschen Wert geteilt! (oder?)







\end{enumerate}