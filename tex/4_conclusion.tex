
\section{Fazit /  Ergebnisse}
\subsection{Auswertung der Typisierungen}
\subsubsection{Konfusionen}
Bei der �berpr�fung der Vertr�glichkeit zweier Markierungen gibt es immer wieder den Fall, dass Gutachter A den Typ X, Gutachter B hingegen den Typ Y f�r sinnvoller h�lt. Daher wurde �berpr�ft, ob es Typen gibt, die besonders h�ufig mit einander verglichen werden. Mit Hilfe der folgenden SQL-Anweisung wurden alle aufgetretenen Konfusionen inklusive H�ufigkeit des Auftretens ausgegeben: 
\begin{lstlisting}[language=SQL] 
SELECT atype_id, btype_id, COUNT(*) 
FROM extractor_confusions 
GROUP BY atype_id, btype_id 
ORDER BY COUNT(*) DESC;
\end{lstlisting} 
Wenn jetzt noch die Reihenfolge ignoriert wird, also die Ergbenisse der Tupel (X,Y) mit denen von (Y,X) addiert werden, erh�lt man folgende Konfusionen, absteigend sortiert nach H�ufigkeiten des Auftretens. 
\begin{figure}[H]
\centering
\begin{tabular}{|c|c|c|c|}\hline
  Nr. & Typ A & Typ B & Anzahl \\ \hline \hline
  1 &Functionality and Behaviour      &    Structure and Relationships    &  439 \\ \hline
  2 &Functionality and Behaviour      &    Purpose and Rationale            &  271 \\ \hline
  3 & Functionality and Behaviour      &    Concepts                                &  243 \\ \hline
  4 &Functionality and Behaviour      &    Directives                               &  205 \\ \hline
  5 &Functionality and Behaviour      &    Non-Information                     &  177 \\ \hline
  6 & Functionality and Behaviour      &    Qual. Attributes, Intern. Aspects    &  176 \\ \hline
  7 & Functionality and Behaviour      &    Environment                           &  135 \\ \hline
  8 & Functionality and Behaviour      &    Patterns                                  &  107 \\ \hline
  9 &  Concepts                                  &    Structure and Relationships     &  105 \\ \hline
  10 & Purpose and Rationale               &    Structure and Relationships   &  90 \\ \hline
  11 & Patterns                                     &    Code Examples                      &  86 \\ \hline
  12 & Functionality and Behaviour       &    Control-Flow                          &  83 \\ \hline
  13 & Concepts     								&   Control-Flow 							& 79 \\ \hline
  14 &  Qual. Attributes, Intern. Aspects & Structure and Relationships & 70 \\ \hline
  15 & Purpose and Rationale 				& Patterns 									 & 67 \\ \hline
  16 & Structure and Relationships     & Patterns 										& 65 \\ \hline
  17 & Functionality and Behavior 			&  Code Examples 						& 63 \\ \hline
  18 &  Qual. Attributes, Intern. Aspects & Environment 					 & 60 \\ \hline
  19 & Concepts 									&  Qual. Attributes, Intern. Aspects & 59 \\ \hline
   20 & Directives 									& Structure and Relationships & 53 \\ \hline
   21 & Concepts 									& Patterns 								& 51 \\ \hline
 \end{tabular}
\caption[Konfusionsh�ufigkeiten]{Konfusionsh�ufigkeiten}
\end{figure} 
Aufgelistet werden die Konfusionen hier, sobald die untere Schranke von 50 Vorkommen erreicht wird. Um sich sp�ter leichter auf einzelne Konfusionen beziehen zu k�nnen, wurden diese durchnummeriert.
\subsection{Herausforderungen}