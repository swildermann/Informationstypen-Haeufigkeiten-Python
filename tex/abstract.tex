\section*{Zusammenfassung}
Walid Maalej und Martin P. Robillard ver�ffentlichten im September 2013 einen Artikel \cite{MaalejRobillard}, in dem sie die Dokumentationen der Programmiersprachen Java und .NET auf ihren Informationsgehalt hin untersucht und verglichen haben. Diese Untersuchung wird im Rahmen dieser Bachelorarbeit auf die Dokumentation von \href{https://www.python.org/}{Python} mit einigen Abweichungen �bertragen. Die von Maalej und Robillard eingef�hrte Taxonomie der in Dokumentationen anzutreffenden Wissenstypen wurde hierf�r auf die Eigenheiten von Python angepasst. Die Einordnung von Teilen der Dokumentationen zu den verschiedenen Wissenstypen wird von Gutachtern im Rahmen eines Forschungspraktikums geleistet und erfolgt mit Hilfe eines eigens hierf�r geschriebenen Werkzeuges. Dieses wurde mit Hilfe des auf Python basierenden Webframeworks \href{https://www.djangoproject.com/}{Django} umgesetzt. F�r die Aufteilung der HTML-Gesamtdokumentation in kleinere Teile und den Import in die Datenbank wurde ein Python-Skript angefertigt, welches f�r die Syntaxanalyse das Paket \href{http://www.crummy.com/software/BeautifulSoup/}{BeautifulSoup4} verwendet. Die statistische Auswertung erfolgte mit \href{http://www.r-project.org/}{R}. 